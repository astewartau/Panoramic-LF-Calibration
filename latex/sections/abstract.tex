\documentclass[../main.tex]{subfiles}
 
\begin{abstract}

Light field cameras are an emerging technology with unique post-processing capabilities. Their applications in computer vision are numerous, especially in robotics, navigation systems and surveillance. Within the medical field, there is a serious potential for occlusion removal in endoscopic imaging and surgical theatre videos using light field cameras. In order for results to be achieved in any such application, calibration is essential. Existing calibration procedures are either conceptually complex metric procedures, or non-metric procedures inflexible to camera orientation and planarity. We present a novel and conceptually simple non-metric calibration for light field acquisition, suitable for camera arrays with constant yet arbitrary camera poses. We also provide a quantitative measure of calibration accuracy, and use it to demonstrate the procedure's efficacy with our Raspberry Pi camera array. Additionally, we present qualitative results by rendering light fields at varying levels of focus and occlusion, and demonstrate success in capturing and rendering light field video - an area of particular interest for our applications. Finally, we reflect on implementation challenges and lessons learned.

\end{abstract}
\newpage
